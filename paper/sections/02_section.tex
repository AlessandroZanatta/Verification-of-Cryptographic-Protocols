% Section 2
% 2021-08-19
% Alessandro Zanatta

\section{Tools description}
\label{section:foundations}

In this section we are going to see a brief overview on the syntax and foundations of Tamarin, Proverif and Verifpal - the three tools we are going to compare.


\subsection{Tamarin prover}
Let us start with Tamarin prover. For a more
in-depth description and further information, see \cite{TamarinFoundations, TamarinFoundationsExtended, TamarinManual}.

The security property model of Tamarin is based on \textbf{labelled multiset rewriting rules} to specify protocols and adversary capabilities, a \textbf{guarded fragment\footnote{The guarded fragment used by Tamarin is basically a subset of formulas from the first order logic with additional constraints on the arguments. See \cite{FragmentFirstOrderLogicPaper} for a definition from a mathematical point of view.} of first order logic} to specify security properties and functions and \textbf{equational theories} to model the algebraic properties of cryptographic protocols \cite{TamarinFoundations}. Additionally, every event in the security properties is annotated with a timepoint $t \in \Q$ and basic comparison of timepoints can be used.

Tamarin then applies a constraint-solving algorithm based on \textbf{backward-search and heuristics} which tries to validate or falsify security properties.

\subsubsection{Multiset rewriting system and state transitions}

\paragraph{Multiset rewriting ingredients} The ingredients of Tamarin multiset rewriting system are the following:

\lstset{language=tamarin}
\begin{itemize}
    \item{\textbf{Terms} - which can be thought of as messages;}
    \item{\textbf{Facts} - which model information in the protocol and are composed by terms. Two built-in facts are used to model network I/O: \lstinline{In(x)} and \lstinline{Out(x)}. A third one, \lstinline{K(x)} is used to model attacker knowledge;}
    \item{\textbf{State of the system} - which is represented using a \textit{multiset} of facts;}
    \item{\textbf{Transition rules} - which defines the possible transitions from one state to another. Multiset rewriting rules are written as $\msr{L}{A}{R}$, where $L$, $A$ and $R$ are multisets of facts, respectively called premises, actions and conclusions;}
    \item{\textbf{Trace} - a sequence $\left<A_1, \dots, A_n\right>$ of sets of ground facts (i.e. facts which do not contain any variable) denoting the sequence of \textit{actions} that happened during a protocol execution.}
\end{itemize}

\paragraph{State transitions}
Consider a ground multiset rewriting rule $\msr{l}{a}{r}$. If we call the state of the system $S$, the trace $T$ and assuming $l \msrsubseteq S$, the new state of the system $S'$ can be defined as $S' = S \msrsetminus l \msrcup r$. Additionally, we append $a$ to the end of the current trace $T$. \textsuperscript{\#} denotes multisets' set operations.





\subsection{Proverif}
\begin{figure}[t]
    \includegraphics[scale=.8]{proverif-verification-method}
    \centering
    \caption{Proverif verification method.\\Inspired by a representation from Bruno Blanchet \cite{SymbolicComputationalBlanchet}.}
    \label{fig:proverif-verification-method}
\end{figure}
Proverif protocols and security properties are based on an extended version of the \pic (the \textbf{applied \picnospace}). The tool also allows the user to define \textbf{constructors, destructors and equations}\footnote{Destructors are basically used to de-construct some previously constructed term (e.g. decryption of an encrypted ciphertext), while equations represent term equality of some sort (e.g. commutativity of multiplication).}, which form the cryptographic primitives. The protocol is then automatically translated to a \textbf{set of Horn-clauses}. Using this abstract representation of the protocol (based on Horn-clauses), the Proverif verifier uses a \textbf{resolution algorithm} on such clauses that allows for verification of security properties \cite{SymbolicComputationalBlanchet}.
A graphical representation of the entire process is given in \cref{fig:proverif-verification-method}.

A brief definition of the grammar of processes of the applied \pic is given in \cref{lst:apic-processes}, slightly modified to match the one actually used by Proverif.

\lstset{language=proverif}
\begin{lstlisting}[caption={Applied \pic grammar of processes.}, label={lst:apic-processes}]
0                    (* null process *)
out(N, M); P         (* output to channel N the message M *)
in(N, M: T); P       (* input from channel N of message M of sort T *)
P | Q                (* parallel composition *)
!P                   (* infinite replication *)
new a: T; P          (* fresh value of sort T *)
if M then P else Q   (* conditional*)
\end{lstlisting}

\lstset{language=proverif}
The null process \lstinline{0} does nothing;
\lstinline{out(N, M); P} (\lstinline{in(N, M: T); P}) outputs (gets) the message \lstinline{M} (of sort \lstinline{T}) into (from) channel \lstinline{N} and then continues with process \lstinline{P};
\lstinline{P | Q} is the parallel composition of \lstinline{P} and \lstinline{Q};
The process \lstinline{!P} effectively behaves as an infinite number of copies of \lstinline{P} running in parallel (\textit{unbounded} replication);
\lstinline{new a: T; P} creates a new fresh value of sort \lstinline{T}, before proceeding with process \lstinline{P};
\lstinline{if M then P else Q} is a standard conditional.

\lstset{language=proverif}
There are many additions to this grammar, such as:
\begin{itemize}
    \item{\lstinline{event EventName(...);} - allows to define a trace of events on which security properties can be defined;}
    \item{\lstinline{query event(EventName(...)).} - queries are used to define security properties. The reserved word \lstinline{attacker(x)} allows to ask Proverif if the attacker knows the term \lstinline{x};}
    \item{\lstinline{phase t;} - allows to execute a process only after processes of phases \lstinline{< t} have been executed. Intuitively, \lstinline{t} can be thought of as a global clock and a process in phase \lstinline{t}  is active only during phase \lstinline{t}. Processes without an explicit phase belong to phase 0;}
    \item{\lstinline{let MacroName = P.} - allows to create a process macro, which is simply substituted when using the name \lstinline{MacroName}.}
\end{itemize}


\begin{figure}[t]
    \makebox[\textwidth][c]{\includegraphics[scale=.55]{verifpal-internals}}
    \centering
    \caption{Verification process of Verifpal.\\ All credits to Nadim Kobeissi \cite{VerifpalManual}.}
    \label{fig:verifpal-verification}
\end{figure}


\subsection{Verifpal}


Let us start with Verifpal syntax by explaining the minimal working example of \cref{lst:verifpal-minimal}.

\lstset{language=verifpal}
\begin{lstlisting}[numbers=left,caption={Verifpal's minimal working example}, label={lst:verifpal-minimal}]
attacker [active]

principal Alice []
principal Bob []

queries []
\end{lstlisting}

The attacker can either be \lstinline{active} or \lstinline{passive}.
Principals are the honest parties of the protocol. Between square brackets many constructs can be used, such as \lstinline{generates}, \lstinline{knows}, assignments and cryptographic primitives.
Finally, in the query block we express security properties we want to check. Queries in Verifpal are limited to pre-defined ones: confidentiality, authentication, freshness, unlinkability and equivalence.
Verifpal also supports phases, like Proverif.


Verifpal does not support user-defined cryptographic primitives. Instead, it supports most of the conventional equational theories, such as Diffie-Hellman, (a)symmetric encryption, authenticated encryption, signatures, blinding and many more.

\subsubsection{Verifpal solving algorithm}

\Cref{fig:verifpal-verification} shows the process of protocol verification used by Verifpal. Let us briefly describe the 5 main steps:

\begin{enumerate}
    \item{\textbf{Gather values} - first of all, the attacker passively observes a protocol execution and gathers all the values shared on the public channel between parties;}
    \item{\textbf{Populate attacker state} - gathered values are inserted into the attacker's state;}
    \item{\textbf{Apply transformations} - the attacker applies the four transformations on all of the obtained value (see bottom-right squares);}
    \item{\textbf{Prepare next mutations} - if the attacker has learnt new values, it creates a combinatorial table of all possible substitutions and uses it to derive a set of possible value substitutions;}
    \item{\textbf{Mutate protocol executions} - finally, the attacker proceeds to execute the protocol, each time applying a mutation from the previous step. As long as the attacker learns new values, it returns to step 1.}
\end{enumerate}

Verifpal, after each stage, checks if any defined security property has been falsified (e.g. the attacker state contains a certain term).
Notice that many other techniques and heuristics are used to avoid state space explosion (e.g. a space state too large to explore in reasonable time).

