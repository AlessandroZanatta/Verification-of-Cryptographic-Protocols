% Section 2
% 2021-08-19
% Alessandro Zanatta

\section{Foundations}
\label{section:foundations}

In this section we are going to see a brief overview on the foundations of Tamarin-prover, Verifpal  and Proverif $-$ the three tools we are going to compare.


\subsection{Tamarin-prover}
Let us start with Tamarin\footnote{We will refer to Tamarin-prover as Tamarin for brevity.}. For a more
in-depth description and further information, see the Tamarin foundations paper \cite{TamarinFoundations} or the extended foundations paper \cite{TamarinFoundationsExtended}.

The security property model of Tamarin is based on labelled multiset rewriting rules to specify protocols and adversary capabilities, a guarded fragment\footnote{The guarded fragment used by Tamarin is basically a subset of formulas from the first order logic with additional constraints on the arguments. See \cite{FragmentFirstOrderLogicPaper} for a definition from a mathematical point of view.} of first order logic to specify security properties\footnote{These are known as \textit{lemmas} in Tamarin.} and functions and equational theories to model the algebraic properties of cryptographic protocols \cite{TamarinFoundations}.

Tamarin then applies a novel constraint-solving algorithm based on heuristics which tries to validate or falsify security properties.

\subsubsection{Multiset rewriting system}
The ingredients of Tamarin's multiset rewriting system are the following:

\begin{itemize}
    \item{Terms $-$ which can be thought of as messages;}
    \item{Facts $-$ which model information in the protocol and are composed by terms;}
    \item{State of the system $-$ which is represented using a \textit{multiset} of facts;}
    \item{Transition rules $-$ which defines the possible transitions from one state to another. We will use the following syntax: $\msr{L}{A}{R}$, where $L$, $A$ and $R$ are multisets of facts, respectively called premises, actions and conclusions;}
    \item{Trace $-$ a sequence $\left<A_1, \dots, A_n\right>$ of sets of ground facts (i.e. facts which do not contain any variable) denoting the sequence of \textit{actions} (or \textit{events}) that happened during the protocol execution.}
\end{itemize}

\subsubsection{Constraint-solving procedure}
\Cref{pseudocode:tamarin-solving-procedure} shows an high level view of the constraint-solving procedure of Tamarin. 

\begin{algorithm}
    \begin{algorithmic}
        \Function{Solve}{\textphi}
        \EndFunction
    \end{algorithmic}
    \caption{Tamarin's constraint solving procedure}
    \label{pseudocode:tamarin-solving-procedure}
\end{algorithm}

\subsection{Verifpal}
We will examine the analysis methodology

\subsection{Proverif}