% Main document
% 2021-08-19
% Alessandro Zanatta

% ---------------------------- %
% Preamble - included packages %
% ---------------------------- %


\documentclass[a4paper,11pt]{article} % Add option 'twoside' only if this document needs to be printed
\usepackage[utf8]{inputenc}
\usepackage[T1]{fontenc}
\usepackage[english]{babel} % Language (last is default)
\usepackage[hidelinks]{hyperref} % Hide red border around links
% \usepackage{FiraMono}
\usepackage{authblk} % Authors affiliations
\usepackage{graphicx} % Images
\usepackage{url}
\graphicspath{{logos/},{images/}} % Images folder(s)
\usepackage{float} % Image positioning
\usepackage{mathtools}
\usepackage{amsfonts}
\usepackage{amsmath}
\usepackage{amssymb}
\usepackage{amsthm} % Proofs
\usepackage{listings} % Listings
\usepackage{boldline} % Thickness of table lines
\usepackage{multirow} % Text of a table cell on multiple rows
\usepackage{longtable} % Tables that span multiple pages
\usepackage{rotating} % Text rotations
\usepackage[fixlanguage]{babelbib} % Bibliography
\bibliographystyle{babplain-fl}
\usepackage[nottoc,notlot,notlof]{tocbibind}

% Authors in italian
\renewcommand\Authsep{, }
\renewcommand\Authand{ e }
\renewcommand\Authands{ e }

% Style
\setcounter{tocdepth}{2} % ToC max depth

% ------------------- %
% Additional packages %
% ------------------- %
\usepackage{cleveref}
\usepackage{afterpage}
\usepackage{msc}
\usepackage{algorithm}
\usepackage{algpseudocode}
\usepackage[euler]{textgreek}

% Theorems
\theoremstyle{plain}
\newtheorem{theorem}{Theorem}[section] % Theorem
\newtheorem{lemma}[theorem]{Lemma} % Lemma
\newtheorem{corollary}{Corollary}[theorem] % Corollary
\theoremstyle{definition}
\newtheorem{definition}[theorem]{Definition} % Definition
\newtheorem{example}[theorem]{Example} % Example

% Short names and commands
\newcommand{\email}[1]{\href{mailto:#1}{\footnotesize\texttt{#1}}} % Email
\newcommand{\e}[1]{\times 10^{#1}} % Scientific notation
\newcommand{\cmark}{\ding{51}} % Checkmark
\newcommand{\xmark}{\ding{55}} % Xmark

% ------------------ %
% ---- Commands ---- %
% ------------------ %

% Comments
\newcommand{\comment}[1]{}

% msc options command
\newcommand\setmscoptions{%
  \setlength{\instdist}{3cm}%
  % \setlength{\levelheight}{1.5 \levelheight}%
  % \setlength{\instwidth}{3cm}
  \setmsckeyword{}
  \drawframe{no}
  \centering
}

\newcommand*{\Z}{\ensuremath{\mathbb{Z}}}

% msr
%% Taken from https://hal.inria.fr/file/index/docid/955869/filename/sapic.tex
\newcommand{\msrewrite}[1]{\mathrel{-\hspace{-2pt}[#1]\hspace{-4pt}\to}}
\newcommand{\emptyrule}{\ensuremath{[]}\xspace}
\newcommand{\msr}[3]{\ensuremath{#1 \msrewrite{#2} #3}}
%% -------------- %%

\newcommand{\msrnolabel}[2]{\ensuremath{#1 \rightarrow #2}}
\newcommand{\msrsetminus}{\ensuremath{\setminus^\#}}
\newcommand{\msrcap}{\ensuremath{\cap^\#}}
\newcommand{\msrcup}{\ensuremath{\cup^\#}}
\newcommand{\msrin}{\ensuremath{\in^\#}}
\newcommand{\msrsubseteq}{\ensuremath{\subseteq^\#}}

% ------------------ %
% Customizable stuff %
% ------------------ %

% Thesis title and metadata
\def\thetitle{Comparison of Tools for the Verification of Cryptographic Protocols}
\def\subtitle{A practical exploratory approach to three different tools: Tamarin-prover, Verifpal and Proverif}
\author[a]{Alessandro Zanatta}
\renewcommand\Affilfont{\small}
\def\uni{University\\of Udine} % University name
\def\course{Computer Network Security} % Course name
\def\ay{2020/21} % Academic year

% Document
\begin{document}

    % Title
    ~\vspace{2.5em}

\begin{flushleft}
    % UniUd logo
    \begin{minipage}{0.1\textwidth}
        \includegraphics[width=0.9\textwidth]{uniud}
    \end{minipage}
    % University name
    \begin{minipage}{0.4\textwidth}
        \textsc{\uni}
    \end{minipage}
    \hfill
    % Academic year and course name
    \begin{minipage}{0.4\textwidth}
        \begin{flushright}
            Academic year \ay\\
            Course of \course
        \end{flushright}
    \end{minipage}
\end{flushleft}

\vspace{-5pt}

% Title
\begin{center}
    \LARGE
    \thetitle\\
    \vspace{0.5em}
    \normalsize
    \subtitle
\end{center}

\vspace{-5pt}

% Authors
\begin{center}
    \makeatletter
    \@author
    \makeatother
\end{center}

\vspace{-5pt}

    % Abstract
    % Abstract
% 2021-08-19
% Alessandro Zanatta

\begin{abstract}
    Cryptographic protocols and algorithms are nowadays widely used to obtain a secure communication over an insecure network. Many common-use technologies have demanding security and privacy requirements. Correctness of protocols deployed for this applications has been proven to be an hard problem for designers. Automated tools can help the designer improve the reliability and security of protocols. There are many publicly available free tools, each with its own strengths and weaknesses. In this short paper we will compare three of them: Tamarin-prover, Proverif and Verifpal.
\end{abstract}

    % Table of contents
    \tableofcontents

    % Body
    \newpage
    % Section 1
% 2021-08-19
% Alessandro Zanatta

\section{Introduction}
\label{section:introduction}

Security protocols are used everyday by billions of users and applications to guarantee a certain degree of security and privacy over communications happening on the (insecure) Internet (MTProto2.0 \cite{Telegram-MTProto2.0}, SSH \cite{rfc4251} and TLSv1.3 \cite{TLSv1.3_specs} protocols are just a few famous examples). Although, there is a catch: designing such protocols has been proven to be extremely error prone. As an example, consider the Needham-Schroeder public-key protocol \cite{NSPK}, which has been believed to be secure for almost 20 years - before a fatal flaw was found and corrected \cite{NSPK_LoweGavin}.

Given the importance of the correctness of such protocols and the difficulties for designers to ensure it, it has been necessary to \textit{formally} prove the absence of security vulnerabilities. Towards this aim, a set of tools have been developed to assist the designing process of a new protocol.

\subsection{Classification of tools}

Two classes of models employed by these tools can be identified: symbolic and computational. As the tools we are going to examine exploit the symbolic model, we will not explore the computational model (see \cite{ReconcilingComputationalSymbolic, SymbolicComputationalBlanchet, 10.1007/978-3-540-31987-0_12} for further readings). 

Let us briefly examine the symbolic model. This model assumes a Dolev-Yao \cite{Dolev-Yao} attacker. Moreover, cryptographic primitives are considered as black-box and are represented using function symbols, the messages are terms and the adversary can only use defined primitives. An important aspect to note of this model is that it assumes \textbf{perfect cryptography}. As an example, consider the case in which there are two function symbols (\textit{enc} and \textit{dec}, used to encrypt and decrypt), a message \textit{m} and a key \textit{k} and the following equality is defined:

\begin{equation}
\label{eq:perfect-crypto}
    \mbox{dec}\left(\mbox{enc}\left(m, k\right), k\right) = m
\end{equation}

Following from \cref{eq:perfect-crypto} $-$ and considering the perfect cryptography assumption $-$ it is possible to decrypt $\mbox{enc}\left(m, k\right)$ if and only if \textit{k} is known \cite{SymbolicComputationalBlanchet}.

In the next sections, we will have a look at the internal reasoning, at some examples and at the comparison of three automatic tools: \textbf{Tamarin-prover, Proverif and Verifpal}.\clearpage
    % Section 2
% 2021-08-19
% Alessandro Zanatta

\section{Tools description}
\label{section:foundations}

In this section we are going to see a brief overview on the foundations of Tamarin prover, Verifpal  and Proverif - the three tools we are going to compare.


\subsection{Tamarin prover}
Let us start with Tamarin prover. For a more
in-depth description and further information, see the Tamarin foundations paper \cite{TamarinFoundations} or the extended foundations paper \cite{TamarinFoundationsExtended}.

The security property model of Tamarin is based on labelled multiset rewriting rules to specify protocols and adversary capabilities, a guarded fragment\footnote{The guarded fragment used by Tamarin is basically a subset of formulas from the first order logic with additional constraints on the arguments. See \cite{FragmentFirstOrderLogicPaper} for a definition from a mathematical point of view.} of first order logic to specify security properties and functions and equational theories to model the algebraic properties of cryptographic protocols \cite{TamarinFoundations}. Additionally, every event in the security properties is annotated with a timepoint $t \in \Q$ and basic comparison of timepoints can be used.

Tamarin then applies a constraint-solving algorithm based on backward-search and heuristics which tries to validate or falsify security properties.

\subsubsection{Multiset rewriting system}
The ingredients of Tamarin multiset rewriting system are the following:

\begin{itemize}
    \item{\textbf{Terms} - which can be thought of as messages;}
    \item{\textbf{Facts} - which model information in the protocol and are composed by terms;}
    \item{\textbf{State of the system} - which is represented using a \textit{multiset} of facts;}
    \item{\textbf{Transition rules} - which defines the possible transitions from one state to another. We will use the following syntax: $\msr{L}{A}{R}$, where $L$, $A$ and $R$ are multisets of facts, respectively called premises, actions and conclusions;}
    \item{\textbf{Trace} - a sequence $\left<A_1, \dots, A_n\right>$ of sets of ground facts (i.e. facts which do not contain any variable) denoting the sequence of \textit{actions} that happened during a protocol execution.}
\end{itemize}

\comment{
    \subsubsection{Constraint-solving procedure}
    \Cref{pseudocode:tamarin-solving-procedure} shows an high level view of the constraint-solving procedure of Tamarin. This solving procedure is based on a backward-search algorithm.

    \begin{algorithm}
        \begin{algorithmic}
            \Function{Solve}{\textphi}
            \EndFunction
        \end{algorithmic}
        \caption{Tamarin's constraint solving procedure}
        \label{pseudocode:tamarin-solving-procedure}
    \end{algorithm}
}

\subsection{Verifpal}

% TODO explain a bit of the syntax of verifpal!!!!

Verifpal also supports phases like Proverif (of course, with a different syntax).

\begin{figure}[t]
    \makebox[\textwidth][c]{\includegraphics[scale=.6]{verifpal-internals}}
    \centering
    \caption{Verification process of Verifpal.\\ All credits to Nadim Kobeissi \cite{VerifpalManual}.}
    \label{fig:verifpal-verification}
\end{figure}

\Cref{fig:verifpal-verification} shows the process of protocol verification used by Verifpal. Let us describe the 5 main steps:

\begin{enumerate}
    \item{\textbf{Gather values} - first of all, the attacker passively observes a protocol execution and gathers all the values shared on the public channel between parties;}
    \item{\textbf{Populate attacker state} - gathered values are inserted into the attacker's state;}
    \item{\textbf{Apply transformations} - the attacker applies the four transformations on all of the obtained value (see bottom-right squares in \cref{fig:verifpal-verification});}
    \item{\textbf{Prepare next mutations} - if the attacker has learnt new values due to the transformations, it creates a combinatorial table of all possible substitutions and use it to derive a set of possible value substitutions across future executions of the protocol;}
    \item{\textbf{Mutate protocol executions} - finally, the attacker proceeds to execute the protocol, each time applying a mutation from the previous step. As long as the attacker keeps on learning new values, this entire process of gathering, populating, transforming, preparing and mutating is repeated.}
\end{enumerate}

Verifpal, after each stage, checks if any defined security property has been falsified (e.g. the attacker state contains a certain message).

\subsection{Proverif}


\begin{figure}[t]
    \includegraphics[scale=.8]{proverif-verification-method}
    \centering
    \caption{Proverif verification method.\\Inspired by a representation from Bruno Blanchet \cite{SymbolicComputationalBlanchet}.}
    \label{fig:proverif-verification-method}
\end{figure}

Proverif protocols and security properties are based on an extended version of the \pic (the \textit{applied} \picnospace). The tool also allows the user to define constructors, destructors and equations\footnote{Destructors are basically used to de-construct some previously constructed term (e.g. decryption of an encrypted ciphertext), while equations represent term equality of some sort (e.g. commutativity of multiplication).}, which form the cryptographic primitives. The protocol is then automatically translated to a set of Horn-clauses. Using this abstract representation of the protocol (based on Horn-clauses), the Proverif verifier uses a resolution algorithm on such clauses that allows for verification of security properties \cite{SymbolicComputationalBlanchet}.
A graphical representation of the whole process is given in \cref{fig:proverif-verification-method}.

A brief definition of the grammar of processes of the applied \pic is given in \cref{eq:apic-processes}. This syntax is identical to the one actually used by Proverif.

\lstset{language=proverif}
\begin{lstlisting}
0                  (* null process *)
out(N, M); P       (* output to channel N the message M *)
in(N, M: T); P     (* input from channel N of message M of sort T *)
P | Q              (* parallel composition *)
!P                 (* infinite replication *)
new a: T; P        (* fresh value of sort T *)
if M then P else Q (* conditional*)
\end{lstlisting}

\lstset{language=proverif}
The null process \lstinline{0} does nothing;
\lstinline{out(N, M); P} (\lstinline{in(N, M: T); P}) outputs (gets) the message \lstinline{M} (of sort \lstinline{T}) into (from) channel \lstinline{N} and then continues with process \lstinline{P};
\lstinline{P | Q} is the parallel composition of \lstinline{P} and \lstinline{Q};
The process \lstinline{!P} effectively behaves as an infinite number of copies of \lstinline{P} running in parallel (\textit{unbounded} replication);
\lstinline{new a: T; P} creates a new fresh value of sort \lstinline{T}, before proceeding with process \lstinline{P};
\lstinline{if M then P else Q} if a standard conditional.

\lstset{language=proverif}
There are many additions to this grammar, such as:
\begin{itemize}
    \item{\lstinline{event EventName(x, y);} - allows to define a trace of events on which security properties can be defined;}
    \item{\lstinline{query event(EventName(x, y))} - queries are used to define security properties. The reserved word \lstinline{attacker(x)} allows to ask Proverif if the attacker knows the term $x$;}
    \item{\lstinline{phase t;} - allows to execute a process only after processes of phases $< t$ have been executed. Intuitively, $t$ can be thought of as a global clock and a process in phase $t$ is active only during phase $t$.}
\end{itemize}\clearpage
    % Section 3
% 2021-08-19
% Alessandro Zanatta

\section{Case study}
\label{section:case-study}

The full models of the following examples are available on github \cite{CaseStudies}. All tools obtained the same results, the exception being Verifpal's model for the Needham-Schroeder-Lowe Public Key protocol.

\subsection{Diffie-Hellman key exchange}

We will examine three different models for the Diffie-Hellman key exchange protocol: anonymous, ephemeral and ephemeral with post-compromise of exponents. This allows us to explore many advanced features and modelling techniques of each tool. \Cref{fig:dh-key-exchange} shows a schematic representation of the protocol.

In the anonymous version we model an un-authenticated Diffie-Hellman, where no security property can hold as the adversary is free to perform a man-in-the-middle attack \cite{MITM-DH}.

In the ephemeral version we model a client-server Diffie-Hellman exchange in which the server's half key is authenticated (e.g. by an X.509 certificate). The server is willing to execute the protocol with any client, but the server is always authenticated.

Finally, we model a Diffie-Hellman with post-compromise of ephemeral keys. This is how Perfect Forward Secrecy \cite{PFS} is usually tested in the symbolic model. Of course, secrecy of the exchanged messages will not hold.

\begin{figure}[t]
    \setmscoptions
    \begin{msc}{}
        \setmscscale{.7}

        \declinst{client}{}{Client}
        \declinst{server}{}{Server}

        \action*{\parbox{3.5cm}{\centering
                Knows $g, p$\\
                $c \in \Z$\\
                $g_c := \modexp{g}{c}{p}$
            }}{client}

        \nextlevel[5]
        \mess{$g_c$}{client}{server}
        \nextlevel

        \action*{\parbox{3.5cm}{\centering
                Knows $g, p$\\
                $s \in \Z$\\
                $g_s := \modexp{g}{s}{p}$\\
                $\skey{c} := \modexp{g_c}{s}{p}$
            }}{server}

        \nextlevel[6]
        \mess{$g_s$}{server}{client}
        \nextlevel

        \action*{\parbox{3.5cm}{\centering
                $\key{cs} := \modexp{g_s}{c}{p}$
            }}{client}
        \nextlevel

    \end{msc}
    \centering
    \caption{Diffie-Hellman key exchange protocol}
    \label{fig:dh-key-exchange}
\end{figure}

\subsubsection{Implementation notes}

We will describe implementation details worth of note in the next paragraphs. Notice that we will use an incremental approach, describing only what was \textit{changed} between different versions.

\paragraph{Anonymous} The implementation of the anonymous version is straightforward. We use two particular constructs for Proverif and Tamarin to store keys: tables and persistent facts, respectively. Both serve the same purpose: store some permanent information (which is not, by default, available to the attacker). The first supports two operations: \textit{insert} and \textit{get}. The latter is, in essence, a \textit{fact} with the additional property of never being removed from the global state.

\paragraph{Ephemeral} In this version, the server half key needs to be authenticated\footnote{An additional requirement is that, while the half key cannot be mutated by the attacker, it must be available to it.}. In Verifpal there is a single (simple) way: guarded variables. Guarded variables (variables between square brackets) are not mutable by the attacker (but known by it). In Tamarin we model this with a simple private channel ruled by a passive attacker. This is done with the following multiset rewriting rule:


\lstset{language=tamarin}
\begin{lstlisting}
rule CertificateExchange:
    [ CertificateOut(x) ] --> [ CertificateIn(x), Out(x) ]
\end{lstlisting}

Finally, in Proverif we model this with a table, additionally making sure to output everything we insert into it. We could have also used a private channel, but that makes the verification of certain queries yield an inconclusive result.

\paragraph{Post-compromise} In Tamarin, we define the following rule (for both client and server):

\lstset{language=tamarin}
\begin{lstlisting}
rule RevealClientEphemeralKey:
    [ ClientEphemeralKey(c) ]
  --[ RevealedClientEphemeralKey(c) ]->
    [ Out(c) ]
\end{lstlisting}

This actually models a \textit{generic} compromise rule. We can exploit timepoints in security properties to assert that the compromise must happen \textbf{after} some other event (e.g. parties have exchanged a message), making this a post-compromise.

In Proverif, we define a process macro which simply outputs the ephemeral key in phase 1:

\lstset{language=proverif}
\begin{lstlisting}
let PostRevealClientEphemeralKey =
    phase 1;

    (* Get the client's ephemeral key and output it *)
    get ClientEphemeralKeyTable(c) in
    out(io, c);
    event PostRevealedClientEphemeralKeyTable(c);
    0.
\end{lstlisting}

\lstset{language=verifpal}
Modeling post-compromise in Verifpal requires using phases again. We can then use Verifpal's built-in construct for revealing terms: \lstinline{leaks x}.
\newpage
\begin{lstlisting}
phase [1]

principal Client [
    leaks c
]

principal Server [
    leaks s
]
\end{lstlisting}

\subsection{Needham-Schroeder Public Key protocol}
\label{sec:NSPK}

\begin{figure}[t]
    \setmscoptions
    \begin{msc}{}
        \setmscscale{.7}

        \declinst{alice}{}{Alice}
        \declinst{bob}{}{Bob}

        \action*{\parbox{3.5cm}{\centering
                Knows $\skey{A}, \pkey{A}$\\
                Knows $\pkey{B}$
            }}{alice}

        \action*{\parbox{3.5cm}{\centering
                Knows $\skey{B}, \pkey{B}$\\
                Knows $\pkey{A}$
            }}{bob}
        \nextlevel[4]

        \action*{\parbox{3.5cm}{\centering
                Generates $N_A$
            }}{alice}

        \nextlevel[3]
        \mess{$\enc{N_A, A}{\pkey{B}}$}{alice}{bob}
        \nextlevel


        \action*{\parbox{3.5cm}{\centering
                Generates $N_B$
            }}{bob}

        \nextlevel[3]
        \mess{$\enc{N_A, N_B}{\pkey{A}}$}{bob}{alice}
        \nextlevel[2]
        \mess{$\enc{N_B}{\pkey{B}}$}{alice}{bob}

    \end{msc}

    \centering
    \caption{Simplified Needham-Schroeder Public Key protocol}
    \label{fig:NSPK}
\end{figure}

The next case-study is the Needham-Schroeder Public Key protocol. Two versions are considered: the flawed version and the fixed one (by Lowe \cite{NSPK_LoweGavin}). A schematic representation of the flawed protocol is shown in \cref{fig:NSPK}, which assumes that clients already know each other's public keys. In the fixed version, message 2 is changed from $\enc{N_A, N_B}{\pkey{A}}$ to $\enc{B, N_A, N_B}{\pkey{A}}$.

\subsubsection{Implementation notes}

In order to re-discover the attack, we model the protocol in a way that allows the attacker to decide whom the honest initiator executes the protocol with. The responder, however, only wishes to engage with the honest initiator.

\newpage
\paragraph{Flawed}

In Proverif, to allow the behaviour described above, we let the attacker send the principals to the initiator at the start of its macro:

\lstset{language=proverif}
\begin{lstlisting}
let Initiator() =
(* Attacker sends communicating parties *)
in(io, (X: Principal, rUser: Principal));

(* We restrict them: X must be honest and they must be different *)
let iUser = isHonest(X, Alice, Bob) in
if iUser <> rUser then (* Protocol starts here *)
\end{lstlisting}

In Tamarin we follow a very similar approach, but we define the initiator to be always Alice (i.e. the attacker chooses \textit{only} the responder). Using the same strategy of Proverif would lead to a very complex protocol with termination issues.

In Verifpal the implementation is straightforward, with a single caveat: when pre-sharing public keys we need to guard only Alice key. This allows the attacker to mutate Bob's key to its own.

\lstset{language=verifpal}
\begin{lstlisting} 
Alice -> Bob: [K_PA]
Bob -> Alice: K_PB
\end{lstlisting}


\paragraph{Fixed} No changes to the models were applied but the fix to the Needham-Schroeder Public Key protocol described at the start of \cref{sec:NSPK}.

\subsection{Results}
\begin{figure}[t]
    \centering
    \makebox[\textwidth][c]{\includegraphics{verifpal-nspk-trace}}
    \caption{Verifpal fixed NSPK attack trace.}
    \label{fig:verifpal-nspk-trace}
\end{figure}

All three tools returned the expected result:
\begin{itemize}
    \item{Diffie-Hellman
                \begin{itemize}
                    \item Anonymous: secrecy and authentication of both parties do not hold;
                    \item Ephemeral: secrecy of client messages and authentication of the server hold. Secrecy of server messages does not hold as the server is willing to initiate a run of the protocol with anybody, even the attacker;
                    \item Post-Compromise Ephemeral: secrecy of client messages no longer holds, but messages from the server are still authenticated.
                \end{itemize}
          }
    \item{Needham-Schroeder Public Key
                \begin{itemize}
                    \item Flawed: Secrecy of the nonces and authentication do not hold.
                    \item Fixed: Nonce secrecy and authentication of the initiator hold. However, as the initiator is willing to execute the protocol with anyone, authentication of the responder does not hold.
                \end{itemize}
          }
\end{itemize}

\lstset{language=verifpal}
There is an exception: Verifpal reports the fixed Needham-Schroeder Public Key  protocol as flawed and finds attack traces for both authentication and confidentiality queries.
Inspecting the tool's output shown in \cref{fig:verifpal-nspk-trace} carefully we can notice a strange result: in the penultimate line, the checked primitive \lstinline{ASSERT}\footnote{A checked primitive in Verifpal is a primitive that aborts execution of the protocol run when it fails. A question mark at the end of a primitive indicates that it is checked. The \lstinline{ASSERT} primitive simply checks that both arguments are actually the same term.} fails right after. In reality, an execution with a failed assertion should be immediately aborted, but Verifpal uses such executions to falsify queries.

Moreover, \textbf{this is not a bug} but the intended behaviour. In a mail on the Verifpal's mailing list we can find the Verifpal's author opinion (Nadim Kobeissi) on a similar problem: ``This analysis result appears to be completely correct to me. The ASSERT call occurs after Bob decrypts enca. So yes, the protocol does not finish its run, but Bob still manages to use the non-authentic enca before the protocol is aborted! Hence the result'' \cite{VerifpalMail}.

It does not seem like there is a way of changing this behaviour, but we can indeed confirm that Verifpal is reporting false traces by looking at the penultimate line: if it contains a checked primitive that obviously fails, then the attack trace is incorrect, otherwise there may be an attack. Notice that this does not mean that there is no attack trace for a certain property as the incorrect one might be masquerading a valid one.\clearpage
    % Section 4
% 2021-08-19
% Alessandro Zanatta

\section{Tools comparison}
\label{section:features-comparison}
In this section we will compare tools based on their usability, expressiveness, efficiency, soundness and completeness.

\subsection{Usability}
Let us define usability as the \textbf{easiness of modeling} and specifying security properties of a certain protocol, as well as how easy it is to interpret the \textbf{tool output}, especially when an attack trace is found.

\paragraph{Modeling protocols}
As Verifpal manual also states \cite{VerifpalManual}, ``Verifpal's main aim is to appeal more to real-world practitioners, students and engineers without sacrificing comprehensive formal verification features''. Its language is extremely simple and fast to learn, queries and equational theory, while severely restricted, are usually expressive enough for most cases.
Meanwhile, Proverif and Tamarin both have a steep learning curve as the constructs they are based on can be challenging to grasp at first. Of course, such a statement is hard to prove and has to be partially taken as a personal indication.

Verifpal additionally has a VS-Code extension which enhances the modeling experience with a real-time graph of the protocol being modeled. \Cref{fig:verifpal-protocol-graph} shows the graph for the Anonymous Diffie-Hellman protocol.

\begin{figure}[t]
    \centering
    \makebox[\textwidth][c]{\includegraphics[scale=.45]{verifpal-protocol-graph}}
    \caption{Verifpal's graph using VS-Code plugin.}
    \label{fig:verifpal-protocol-graph}
\end{figure}


\paragraph{Attack traces}
Finally, let us compare tool outputs when an attack trace is found. Verifpal states that its trace is easier to read compared to other tools. Considering a solely textual attack trace in \cref{fig:attack-trace}, it is actually true that Verifpal's trace is easier to understand for an un-experienced user as it indicates what messages were mutated by the adversary.

\begin{figure}[t]
    \centering
    \makebox[\textwidth][c]{\includegraphics[scale=.65]{attack-trace-comparison}}
    \caption{Textual traces comparison.}
    \label{fig:attack-trace}
\end{figure}

However, Tamarin and Proverif also offer a graph view of the attack, which is far more readable. Tamarin traces, which are offered from its interactive web app, are slightly customizable and can be made more or less verbose. An example is given in \cref{fig:tamarin-trace,fig:proverif-trace}.


\begin{figure}[t]
    \centering
    \makebox[\textwidth][c]{\includegraphics[scale=.75]{tamarin-trace}}
    \caption{Tamarin's attack trace graph view example.}
    \label{fig:tamarin-trace}
\end{figure}
\begin{figure}[t]
    \centering
    \makebox[\textwidth][c]{\includegraphics[scale=.3]{proverif-trace}}
    \caption{Proverif's attack trace graph view example.}
    \label{fig:proverif-trace}
\end{figure}


\subsection{Expressiveness}
\Cref{tbl:expressiveness} shows expressiveness of analyzed tools.
All tools assume unbounded sessions. Tamarin's equational theory, with its complex Diffie-Hellman theory (which comprehends 6 equations, including inverse), is way more expressive then the other two tools (in which only commutativity of exponents is possible). Verifpal additionally puts a cap on the number of exponentiations to two, which is not high enough to model Diffie-Hellman with three or more parties \cite{MultipartyDH}.

Proverif support to state is limited to tables, while Tamarin and Verifpal naturally support stateful principals (via different formalisms and syntaxes). Moreover, observational equivalence\footnote{Observational equivalence expresses that two systems appear the same to the environment.} is not supported by Verifpal (which only supports equivalence of terms), but it supports linkability\footnote{Verifpal uses the following definition \cite{VerifpalFoundations}: ``for two
    observed values, the adversary cannot distinguish between a protocol execution in
    which they belong to the same user and a protocol execution in which they belong to
    two different users''.}.

\begin{table}[!ht]
    \renewcommand{\arraystretch}{1.5}
    \setlength\arrayrulewidth{1pt}
    \rowcolors{2}{gray!25}{white}
    \makebox[\textwidth][c]{
        \scalebox{0.9}{
            \begin{tabular}{c|ccccc}
                \multicolumn{1}{l|}{} & \textbf{Unbounded} & \textbf{Equational theory} & \textbf{State} & \textbf{Obs. Equiv.} & \textbf{Linkability} \\ \hline
                \textbf{Tamarin}      & \fullcirc          & \fullcirc                  & \fullcirc      & \fullcirc            & \emptycirc           \\
                \textbf{Proverif}     & \fullcirc          & \halfcirc                  & \halfcirc      & \fullcirc            & \emptycirc           \\
                \textbf{Verifpal}     & \fullcirc          & \halfcirc                  & \fullcirc      & \emptycirc           & \fullcirc
            \end{tabular}
        }
    }
    \caption{Expressiveness of tools.}
    \label{tbl:expressiveness}
\end{table}

\subsection{Efficiency}
\Cref{tab:DH,tab:NSPK} shows three metrics for each tool and protocol version: peak memory usage (in kb), elapsed time since start and cpu time (as system load).

Measures were taken using hyperfine \cite{hyperfine} and time \cite{time}.

As it can be seen, Tamarin memory usage is about 3 to 5 times the one of Proverif and Verifpal, while time to complete proofs is about one to two orders of magnitude higher. We can also notice that Proverif is a single-threaded prover\footnote{As cpu time is always $\leq 1$, this means that Proverif is using at most a single core.}, while Tamarin and Verifpal are extensively using all available cores\footnote{The benchmark system has a total of 4 cores.}.


\begin{table}[!ht]
\setlength\arrayrulewidth{1pt}
\rowcolors{2}{gray!25}{white}
\makebox[\textwidth][c]{
    \scalebox{0.9}{
    \begin{tabular}{c|ccc|ccc|ccc|l}
    \cline{2-10}
    \multicolumn{1}{l|}{} & \multicolumn{3}{c|}{\textbf{Peak memory size (kb)}} & \multicolumn{3}{c|}{\textbf{Time (ms)}} & \multicolumn{3}{c|}{\textbf{CPU time}} &  \\ \cline{2-10}
    \multicolumn{1}{l|}{}                    & Tamarin & Verifpal & Proverif & Tamarin & Verifpal & Proverif & Tamarin & Verifpal & Proverif & \\ \hline
    \multicolumn{1}{c|}{Mean}      & 52445 & 11480 & 10639 & 1136 & 13  & 44  & 3.52 & 2.08 & 0.98 & \multicolumn{1}{c}{} \\ \cline{1-10}
    \multicolumn{1}{c|}{Deviation} & 2541  & 244   & 108   & 106  & 1   & 1   & 0.06 & 0.12 & 0.02 & \multicolumn{1}{c}{} \\ \cline{1-10}
    \multicolumn{1}{c|}{Median}    & 52120 & 11464 & 10616 & 1125 & 12  & 44  & 3.52 & 2.08 & 0.97 & \multicolumn{1}{c}{\parbox[t]{1em}{\multirow{-3}{*}{\rotatebox[origin=c]{90}{\textbf{Anon}}}}} \\ \hline
    
    \multicolumn{1}{c|}{Mean}      & 39731 & 13112 & 10623 & 841  & 67  & 34  & 3.38 & 3.24 & 0.98 & \multicolumn{1}{c}{} \\ \cline{1-10}
    \multicolumn{1}{c|}{Deviation} & 1824  & 257   & 110   & 106  & 7   & 0   & 0.04 & 0.06 & 0.02 & \multicolumn{1}{c}{} \\ \cline{1-10}
    \multicolumn{1}{c|}{Median}    & 39742 & 13008 & 10604 & 814  & 72  & 34  & 3.38 & 3.25 & 0.97 & \multicolumn{1}{c}{\parbox[t]{1em}{\multirow{-3}{*}{\rotatebox[origin=c]{90}{\textbf{Eph}}}}} \\ \hline

    \multicolumn{1}{c|}{Mean}      & 44273 & 13098 & 11112 & 1312 & 75 & 81 & 3.47 & 3.14 & 0.98 & \multicolumn{1}{c}{} \\ \cline{1-10}
    \multicolumn{1}{c|}{Deviation} & 1518  & 250   & 112   & 201  & 8  & 1  & 0.04 & 0.10 & 0.01 & \multicolumn{1}{c}{} \\ \cline{1-10}
    \multicolumn{1}{c|}{Median}    & 43914 & 12994 & 11092 & 1319 & 80 & 81 & 3.48 & 3.16 & 0.98 & \multicolumn{1}{c}{\parbox[t]{1em}{\multirow{-3}{*}{\rotatebox[origin=c]{90}{\textbf{PFS}}}}} \\
    \end{tabular}
    }
}
\caption{Comparison table for Diffie-Hellman.}
\label{tab:DH}
\end{table}

\begin{table}[!ht]
    \setlength\arrayrulewidth{1pt}
    \rowcolors{2}{gray!25}{white}
    \makebox[\textwidth][c]{
        \scalebox{0.6}{
            \begin{tabular}{c|ccc|ccc|ccc|c}
                \cline{2-10}

                \multicolumn{1}{l|}{}          & \multicolumn{3}{c|}{\textbf{Peak memory size (kb)}} & \multicolumn{3}{c|}{\textbf{Time (ms)}} & \multicolumn{3}{c|}{\textbf{CPU time}} &                                                                                                                                                                  \\ \cline{2-10}
                \multicolumn{1}{l|}{}          & Tamarin                                             & Verifpal                                & Proverif                               & Tamarin & Verifpal & Proverif & Tamarin & Verifpal & Proverif &                                                                                                  \\ \hline

                \multicolumn{1}{c|}{Mean}      & 59803                                               & 13286                                   & 11106                                  & 3336    & 40       & 86       & 3.70    & 2.05     & 0.99     & \multicolumn{1}{l}{}                                                                             \\ \cline{1-10}
                \multicolumn{1}{c|}{Deviation} & 2860                                                & 317                                     & 107                                    & 188     & 4        & 1        & 0.05    & 0.05     & 0.01     & \multicolumn{1}{l}{}                                                                             \\ \cline{1-10}
                \multicolumn{1}{c|}{Median}    & 59498                                               & 13224                                   & 11088                                  & 3345    & 40       & 86       & 3.70    & 2.05     & 0.98     & \multicolumn{1}{l}{\parbox[t]{1em}{\multirow{-3}{*}{\rotatebox[origin=c]{90}{\textbf{Flawed}}}}} \\ \hline

                \multicolumn{1}{c|}{Mean}      & 53217                                               & NA                                      & 10698                                  & 1929    & NA       & 69       & 3.63    & NA       & 0.99     & \multicolumn{1}{l}{}                                                                             \\ \cline{1-10}
                \multicolumn{1}{c|}{Deviation} & 2802                                                & NA                                      & 109                                    & 164     & NA       & 1        & 0.05    & NA       & 0.01     & \multicolumn{1}{l}{}                                                                             \\ \cline{1-10}
                \multicolumn{1}{c|}{Median}    & 53400                                               & NA                                      & 10672                                  & 1929    & NA       & 68       & 3.63    & NA       & 0.98     & \multicolumn{1}{l}{\parbox[t]{1em}{\multirow{-3}{*}{\rotatebox[origin=c]{90}{\textbf{Fixed}}}}}  \\
            \end{tabular}
        }
    }
    \caption{Comparison table for Needham-Schroeder Public Key.}
    \label{tab:NSPK}
\end{table}

\newpage
\subsection{Soundness and completeness}

Let us define, in an informal way, what soundness and completeness of formal verification tools means. Please refer to \cite{Furer89oncompleteness} for a more detailed description.

\textbf{Soundness} means that any statement that can be proven is valid (i.e. there is no proof for a false statement).

\textbf{Completeness} means that the proof system is powerful enough to prove any valid statement.

Soundness and completeness of analyzed tools is summarized in \cref{tbl:sound-complete}.
Tamarin is both sound and complete, meaning that it will always give the correct answer to queries, assuming termination. Proverif, being sound but not complete, may be unable to prove some statements and terminate with an inconclusive result. Finally, Verifpal has no formal guarantee at all, but evidence has shown that in real-case scenarios the tool performs correctly.

\begin{table}[!ht]
    \centering
    \setlength\arrayrulewidth{1pt}
    \renewcommand{\arraystretch}{1.4}
    \rowcolors{2}{gray!25}{white}
    \scalebox{0.9}{
        \begin{tabular}{c|cc}
            \multicolumn{1}{l|}{}   & \textbf{Soundness} & \textbf{Completeness} \\ \hline
            \textbf{Tamarin prover} & \cmark             & \cmark                \\
            \textbf{Proverif}       & \cmark             & \xmark                \\
            \textbf{Verifpal}       & \xmark             & \xmark
        \end{tabular}
    }
    \caption{Soundness and completeness of Tamarin, Proverif and Verifpal.}
    \label{tbl:sound-complete}
\end{table}

\comment{
Tamarin
Pros:
- More expressive formulas for security properties
- Sound and complete
- Can prove observational equivalence
- Possibility of manually guiding the proof when heuristics fail to do so automagically
- Can model many algebraic properties of groups for DH key exchanges (comes with higher computational cost!)
- Can model xor and elliptic curve operations
- Supports user defined equational theories
- More flexibility on post compromise properties using timepoints, but lemmas become more verbose
- IS SOUND AND COMPLETE
Cons:
- Probably harder to model
- Slower (do benchmarks!!)


Proverif
Pros:
- Long history
- Can prove observational equivalence
- Supports user defined equational theories
Cons:
- Uses an old-style syntax
- CANNOT model many algebraic properties of groups for DH key exchanges (only commutativity)
- IS SOUND, BUT NOT COMPLETE


Verifpal
Pros:
- Very simple to use (both model specification and queries), while being expressive enough for most use cases (not for 3 parties DH)
- Possibility to translate to proverif/coq (even though its limited)
- Very intuitive language
Cons:
- There may be queries that CANNOT be expressed (try finding a counterexample!)
- CANNOT prove observational equivalence
- It's very recent, still in beta
- CANNOT model many algebraic properties of groups for DH key exchanges (only commutativity ???)
- does NOT support user-defined equational theories
- Does NOT allow to express injectivity
- Does NOT produce a graphical representation of traces automatically
- Feels VERY different from other tools
- Impossible to model 3 parties DH exchange, for example

- IS NOR SOUND NOR COMPLETE


}\clearpage
    % Section 1
% 2021-08-19
% Alessandro Zanatta

\section{Conclusions}
\label{section:conclusions}

Conclusions here\clearpage
    
    % Bibliography
    \bibliography{references}
 % Print all bibliography references

\end{document}